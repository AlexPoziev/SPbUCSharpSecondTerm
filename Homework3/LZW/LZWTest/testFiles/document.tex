\documentclass[a4paper, 14pt]{report}

\usepackage{cmap}
\usepackage[T2A]{fontenc}
\usepackage[utf8]{inputenc}
\usepackage[english, russian]{babel}
\usepackage{amsthm}

\author{Aleksey Poziev}
\title{Аналитическая геометрия}
\date{\today}

\begin{document} %
	\maketitle
\tableofcontents{}
\clearpage

\newtheorem{thm}{Теорема} 
\newtheorem{defin}{Определение}

\chapter{Точечное пространство}

\begin{thm}
	V - векторное пространство. E - множество точек. \\
	Задано E $\times$ V $\rightarrow$ E (сложение)\\
	\begin{enumerate}
		\item (e + $V_1$) + $V_2$ = e + ($V_1$ + $V_2$)
		\item e + $\overline{0}$ = e
		\item $\forall$ $e_1$, $e_2$  $\exists$! V:  $e_1$ + V = $e_2$
	\end{enumerate}
	Если выполняются все условия, то E - называется \textbf{точечным пространством}
\end{thm}
Если мы фиксируем $e_0$, то $\exists$ биекция между E $\leftrightarrow$ V
\begin{itemize}
	\item e $\rightarrow$ $\overline{e - e_0}$ 
	\item $e_0 + \overline{V}$ $\leftarrow$ V
\end{itemize}

\begin{defin}
	$e_0, e_1 \in E$ $|e_0e_1|$ := $|\overrightarrow{e + e_0}|$
\end{defin}
Пусть dim V = 3. i, j, k - ОНБ  в V. $e_0 \in E$ , называется начальной точкой, если $\forall e \: \exists! \V \in V: e = e_0 + V => e = (a, b, c)$
\section{Вторая глава}

	
\end{document}}